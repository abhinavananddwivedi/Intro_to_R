\documentclass[11pt,]{article}
\usepackage{lmodern}
\usepackage{amssymb,amsmath}
\usepackage{ifxetex,ifluatex}
\usepackage{fixltx2e} % provides \textsubscript
\ifnum 0\ifxetex 1\fi\ifluatex 1\fi=0 % if pdftex
  \usepackage[T1]{fontenc}
  \usepackage[utf8]{inputenc}
\else % if luatex or xelatex
  \ifxetex
    \usepackage{mathspec}
  \else
    \usepackage{fontspec}
  \fi
  \defaultfontfeatures{Ligatures=TeX,Scale=MatchLowercase}
\fi
% use upquote if available, for straight quotes in verbatim environments
\IfFileExists{upquote.sty}{\usepackage{upquote}}{}
% use microtype if available
\IfFileExists{microtype.sty}{%
\usepackage{microtype}
\UseMicrotypeSet[protrusion]{basicmath} % disable protrusion for tt fonts
}{}
\usepackage[margin = 1.5in]{geometry}
\usepackage{hyperref}
\PassOptionsToPackage{usenames,dvipsnames}{color} % color is loaded by hyperref
\hypersetup{unicode=true,
            pdftitle={Introduction to RMarkdown},
            pdfauthor={Abhinav Anand},
            colorlinks=true,
            linkcolor=blue,
            citecolor=magenta,
            urlcolor=red,
            breaklinks=true}
\urlstyle{same}  % don't use monospace font for urls
\usepackage{color}
\usepackage{fancyvrb}
\newcommand{\VerbBar}{|}
\newcommand{\VERB}{\Verb[commandchars=\\\{\}]}
\DefineVerbatimEnvironment{Highlighting}{Verbatim}{commandchars=\\\{\}}
% Add ',fontsize=\small' for more characters per line
\usepackage{framed}
\definecolor{shadecolor}{RGB}{248,248,248}
\newenvironment{Shaded}{\begin{snugshade}}{\end{snugshade}}
\newcommand{\KeywordTok}[1]{\textcolor[rgb]{0.13,0.29,0.53}{\textbf{#1}}}
\newcommand{\DataTypeTok}[1]{\textcolor[rgb]{0.13,0.29,0.53}{#1}}
\newcommand{\DecValTok}[1]{\textcolor[rgb]{0.00,0.00,0.81}{#1}}
\newcommand{\BaseNTok}[1]{\textcolor[rgb]{0.00,0.00,0.81}{#1}}
\newcommand{\FloatTok}[1]{\textcolor[rgb]{0.00,0.00,0.81}{#1}}
\newcommand{\ConstantTok}[1]{\textcolor[rgb]{0.00,0.00,0.00}{#1}}
\newcommand{\CharTok}[1]{\textcolor[rgb]{0.31,0.60,0.02}{#1}}
\newcommand{\SpecialCharTok}[1]{\textcolor[rgb]{0.00,0.00,0.00}{#1}}
\newcommand{\StringTok}[1]{\textcolor[rgb]{0.31,0.60,0.02}{#1}}
\newcommand{\VerbatimStringTok}[1]{\textcolor[rgb]{0.31,0.60,0.02}{#1}}
\newcommand{\SpecialStringTok}[1]{\textcolor[rgb]{0.31,0.60,0.02}{#1}}
\newcommand{\ImportTok}[1]{#1}
\newcommand{\CommentTok}[1]{\textcolor[rgb]{0.56,0.35,0.01}{\textit{#1}}}
\newcommand{\DocumentationTok}[1]{\textcolor[rgb]{0.56,0.35,0.01}{\textbf{\textit{#1}}}}
\newcommand{\AnnotationTok}[1]{\textcolor[rgb]{0.56,0.35,0.01}{\textbf{\textit{#1}}}}
\newcommand{\CommentVarTok}[1]{\textcolor[rgb]{0.56,0.35,0.01}{\textbf{\textit{#1}}}}
\newcommand{\OtherTok}[1]{\textcolor[rgb]{0.56,0.35,0.01}{#1}}
\newcommand{\FunctionTok}[1]{\textcolor[rgb]{0.00,0.00,0.00}{#1}}
\newcommand{\VariableTok}[1]{\textcolor[rgb]{0.00,0.00,0.00}{#1}}
\newcommand{\ControlFlowTok}[1]{\textcolor[rgb]{0.13,0.29,0.53}{\textbf{#1}}}
\newcommand{\OperatorTok}[1]{\textcolor[rgb]{0.81,0.36,0.00}{\textbf{#1}}}
\newcommand{\BuiltInTok}[1]{#1}
\newcommand{\ExtensionTok}[1]{#1}
\newcommand{\PreprocessorTok}[1]{\textcolor[rgb]{0.56,0.35,0.01}{\textit{#1}}}
\newcommand{\AttributeTok}[1]{\textcolor[rgb]{0.77,0.63,0.00}{#1}}
\newcommand{\RegionMarkerTok}[1]{#1}
\newcommand{\InformationTok}[1]{\textcolor[rgb]{0.56,0.35,0.01}{\textbf{\textit{#1}}}}
\newcommand{\WarningTok}[1]{\textcolor[rgb]{0.56,0.35,0.01}{\textbf{\textit{#1}}}}
\newcommand{\AlertTok}[1]{\textcolor[rgb]{0.94,0.16,0.16}{#1}}
\newcommand{\ErrorTok}[1]{\textcolor[rgb]{0.64,0.00,0.00}{\textbf{#1}}}
\newcommand{\NormalTok}[1]{#1}
\usepackage{graphicx,grffile}
\makeatletter
\def\maxwidth{\ifdim\Gin@nat@width>\linewidth\linewidth\else\Gin@nat@width\fi}
\def\maxheight{\ifdim\Gin@nat@height>\textheight\textheight\else\Gin@nat@height\fi}
\makeatother
% Scale images if necessary, so that they will not overflow the page
% margins by default, and it is still possible to overwrite the defaults
% using explicit options in \includegraphics[width, height, ...]{}
\setkeys{Gin}{width=\maxwidth,height=\maxheight,keepaspectratio}
\IfFileExists{parskip.sty}{%
\usepackage{parskip}
}{% else
\setlength{\parindent}{0pt}
\setlength{\parskip}{6pt plus 2pt minus 1pt}
}
\setlength{\emergencystretch}{3em}  % prevent overfull lines
\providecommand{\tightlist}{%
  \setlength{\itemsep}{0pt}\setlength{\parskip}{0pt}}
\setcounter{secnumdepth}{0}
% Redefines (sub)paragraphs to behave more like sections
\ifx\paragraph\undefined\else
\let\oldparagraph\paragraph
\renewcommand{\paragraph}[1]{\oldparagraph{#1}\mbox{}}
\fi
\ifx\subparagraph\undefined\else
\let\oldsubparagraph\subparagraph
\renewcommand{\subparagraph}[1]{\oldsubparagraph{#1}\mbox{}}
\fi

%%% Use protect on footnotes to avoid problems with footnotes in titles
\let\rmarkdownfootnote\footnote%
\def\footnote{\protect\rmarkdownfootnote}

%%% Change title format to be more compact
\usepackage{titling}

% Create subtitle command for use in maketitle
\newcommand{\subtitle}[1]{
  \posttitle{
    \begin{center}\large#1\end{center}
    }
}

\setlength{\droptitle}{-2em}
  \title{Introduction to RMarkdown}
  \pretitle{\vspace{\droptitle}\centering\huge}
  \posttitle{\par}
  \author{Abhinav Anand}
  \preauthor{\centering\large\emph}
  \postauthor{\par}
  \date{}
  \predate{}\postdate{}

\linespread{1.25}

\begin{document}
\maketitle

\section{Setup}\label{setup}

The packages \texttt{rmarkdown} and \texttt{knitr} need to be installed
prior to running the commands below. To install, type in the console
\texttt{install.packages(c("knitr",\ "rmarkdown"))}.

\section{Background}\label{background}

A text processor such as MS Word uses a ``What-You-See-Is-What-You-Get''
(WYSIWYG) editor. We see the output of the ``rendered'' text directly.
However, to exercise finer control over the rendering, a ``markup''
language is better. Latex and HTML are examples of markup languages:
they contain ordinary text, as well as special commands that govern the
final rendering of the text. Markdown is an especially convenient markup
language that preserves the finer aspects of text formatting without
being too hard to read.

There are many programs for rendering documents written in Markdown into
documents in the .html, .pdf and .docx formats (among many others). R
Markdown extends Markdown to incorporate text formatting, mathematics,
figures, tables; as well as R code and its output directly into the
rendered document.

Behind the scenes, when we ``knit'' the file, R Markdown sends the
\texttt{.rmd} file to \texttt{knitr()} which executes all code chunks
and creates a new markdown file (extension \texttt{.md}) which includes
both the code and its output. This file is then used by \texttt{pandoc}
(essentially a free and open source document converter) to convert to
the desired format. This two-step workflow can help to create a very
wide range of formatting options for eventual publishing.

\section{The Main Components of
RMarkdown}\label{the-main-components-of-rmarkdown}

Each RMarkdown document has three main components---header, text; and
code chunks.

\subsection{The YAML Header}\label{the-yaml-header}

The current file's header includes the following lines:

\begin{Shaded}
\begin{Highlighting}[]
\OperatorTok{---}
\NormalTok{title}\OperatorTok{:}\StringTok{ "Introduction to RMarkdown"}
\NormalTok{author}\OperatorTok{:}\StringTok{ }\NormalTok{Abhinav Anand}

\NormalTok{output}\OperatorTok{:}
\StringTok{  }\NormalTok{pdf_document}\OperatorTok{:}
\StringTok{    }\NormalTok{keep_tex}\OperatorTok{:}\StringTok{ }\NormalTok{true}

\NormalTok{fontsize}\OperatorTok{:}\StringTok{ }\NormalTok{11pt}
\NormalTok{documentclass}\OperatorTok{:}\StringTok{ }\NormalTok{article}
\NormalTok{geometry}\OperatorTok{:}\StringTok{ }\NormalTok{margin =}\StringTok{ }\FloatTok{1.}\NormalTok{5in}

\NormalTok{linkcolor}\OperatorTok{:}\StringTok{ }\NormalTok{blue}
\NormalTok{urlcolor}\OperatorTok{:}\StringTok{ }\NormalTok{red}
\NormalTok{citecolor}\OperatorTok{:}\StringTok{ }\NormalTok{magenta}

\NormalTok{citation_package}\OperatorTok{:}\StringTok{ }\NormalTok{natbib}
\NormalTok{bibliography}\OperatorTok{:}\StringTok{ }\NormalTok{Working_Paper.bib}

\NormalTok{header}\OperatorTok{-}\NormalTok{includes}\OperatorTok{:}
\StringTok{   }\OperatorTok{-}\StringTok{ }\NormalTok{\textbackslash{}linespread\{}\FloatTok{1.25}\NormalTok{\}}


\OperatorTok{---}
\end{Highlighting}
\end{Shaded}

\subsection{Text}\label{text}

\subsection{Code Chunks}\label{code-chunks}

\subsubsection{Literate Programming and
Reproducibility}\label{literate-programming-and-reproducibility}

\subsubsection{Formatting}\label{formatting}

To insert a break between paragraphs, include a single completely blank
line.

To force a line break, put two blank spaces at the end of a line.

\subsubsection{Headers}\label{headers}

\texttt{\#}, \texttt{\#\#}, \texttt{\#\#\#}

\subsubsection{Italics, Boldface}\label{italics-boldface}

\subsubsection{Backticks}\label{backticks}

\subsubsection{Bullet/Numbered Lists}\label{bulletnumbered-lists}

\subsubsection{Title etc.}\label{title-etc.}

\subsubsection{Links and Images}\label{links-and-images}

\subsubsection{Including Code Chunks}\label{including-code-chunks}

One of the most common options turns off printing out the code, but
leaves the results alone: ```\{r, echo=FALSE\}

Another runs the code, but includes neither the text of the code nor its
output. ```\{r, include=FALSE\} This might seem pointless, but it can be
useful for code chunks which do set-up like loading data files, or
initial model estimates, etc.

Another option prints the code in the document, but does not run it:
```\{r, eval=FALSE\} This is useful if you want to talk about the
(nicely formatted) code.

\subsubsection{Tables}\label{tables}

The default print-out of matrices, tables, etc. from R Markdown is
frankly ugly. The knitr package contains a very basic command, kable,
which will format an array or data frame more nicely for display.

\subsubsection{Math}\label{math}

Just use latex style.

\subsubsection{Bibliography and
Citations}\label{bibliography-and-citations}


\end{document}
