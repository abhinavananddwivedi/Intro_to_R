\documentclass[11pt,]{article}
\usepackage{lmodern}
\usepackage{amssymb,amsmath}
\usepackage{ifxetex,ifluatex}
\usepackage{fixltx2e} % provides \textsubscript
\ifnum 0\ifxetex 1\fi\ifluatex 1\fi=0 % if pdftex
  \usepackage[T1]{fontenc}
  \usepackage[utf8]{inputenc}
\else % if luatex or xelatex
  \ifxetex
    \usepackage{mathspec}
  \else
    \usepackage{fontspec}
  \fi
  \defaultfontfeatures{Ligatures=TeX,Scale=MatchLowercase}
\fi
% use upquote if available, for straight quotes in verbatim environments
\IfFileExists{upquote.sty}{\usepackage{upquote}}{}
% use microtype if available
\IfFileExists{microtype.sty}{%
\usepackage{microtype}
\UseMicrotypeSet[protrusion]{basicmath} % disable protrusion for tt fonts
}{}
\usepackage[margin = 1.5in]{geometry}
\usepackage{hyperref}
\PassOptionsToPackage{usenames,dvipsnames}{color} % color is loaded by hyperref
\hypersetup{unicode=true,
            pdftitle={Introduction to RMarkdown},
            pdfauthor={Abhinav Anand, IIMB},
            colorlinks=true,
            linkcolor=blue,
            citecolor=magenta,
            urlcolor=red,
            breaklinks=true}
\urlstyle{same}  % don't use monospace font for urls
\usepackage{color}
\usepackage{fancyvrb}
\newcommand{\VerbBar}{|}
\newcommand{\VERB}{\Verb[commandchars=\\\{\}]}
\DefineVerbatimEnvironment{Highlighting}{Verbatim}{commandchars=\\\{\}}
% Add ',fontsize=\small' for more characters per line
\usepackage{framed}
\definecolor{shadecolor}{RGB}{248,248,248}
\newenvironment{Shaded}{\begin{snugshade}}{\end{snugshade}}
\newcommand{\KeywordTok}[1]{\textcolor[rgb]{0.13,0.29,0.53}{\textbf{#1}}}
\newcommand{\DataTypeTok}[1]{\textcolor[rgb]{0.13,0.29,0.53}{#1}}
\newcommand{\DecValTok}[1]{\textcolor[rgb]{0.00,0.00,0.81}{#1}}
\newcommand{\BaseNTok}[1]{\textcolor[rgb]{0.00,0.00,0.81}{#1}}
\newcommand{\FloatTok}[1]{\textcolor[rgb]{0.00,0.00,0.81}{#1}}
\newcommand{\ConstantTok}[1]{\textcolor[rgb]{0.00,0.00,0.00}{#1}}
\newcommand{\CharTok}[1]{\textcolor[rgb]{0.31,0.60,0.02}{#1}}
\newcommand{\SpecialCharTok}[1]{\textcolor[rgb]{0.00,0.00,0.00}{#1}}
\newcommand{\StringTok}[1]{\textcolor[rgb]{0.31,0.60,0.02}{#1}}
\newcommand{\VerbatimStringTok}[1]{\textcolor[rgb]{0.31,0.60,0.02}{#1}}
\newcommand{\SpecialStringTok}[1]{\textcolor[rgb]{0.31,0.60,0.02}{#1}}
\newcommand{\ImportTok}[1]{#1}
\newcommand{\CommentTok}[1]{\textcolor[rgb]{0.56,0.35,0.01}{\textit{#1}}}
\newcommand{\DocumentationTok}[1]{\textcolor[rgb]{0.56,0.35,0.01}{\textbf{\textit{#1}}}}
\newcommand{\AnnotationTok}[1]{\textcolor[rgb]{0.56,0.35,0.01}{\textbf{\textit{#1}}}}
\newcommand{\CommentVarTok}[1]{\textcolor[rgb]{0.56,0.35,0.01}{\textbf{\textit{#1}}}}
\newcommand{\OtherTok}[1]{\textcolor[rgb]{0.56,0.35,0.01}{#1}}
\newcommand{\FunctionTok}[1]{\textcolor[rgb]{0.00,0.00,0.00}{#1}}
\newcommand{\VariableTok}[1]{\textcolor[rgb]{0.00,0.00,0.00}{#1}}
\newcommand{\ControlFlowTok}[1]{\textcolor[rgb]{0.13,0.29,0.53}{\textbf{#1}}}
\newcommand{\OperatorTok}[1]{\textcolor[rgb]{0.81,0.36,0.00}{\textbf{#1}}}
\newcommand{\BuiltInTok}[1]{#1}
\newcommand{\ExtensionTok}[1]{#1}
\newcommand{\PreprocessorTok}[1]{\textcolor[rgb]{0.56,0.35,0.01}{\textit{#1}}}
\newcommand{\AttributeTok}[1]{\textcolor[rgb]{0.77,0.63,0.00}{#1}}
\newcommand{\RegionMarkerTok}[1]{#1}
\newcommand{\InformationTok}[1]{\textcolor[rgb]{0.56,0.35,0.01}{\textbf{\textit{#1}}}}
\newcommand{\WarningTok}[1]{\textcolor[rgb]{0.56,0.35,0.01}{\textbf{\textit{#1}}}}
\newcommand{\AlertTok}[1]{\textcolor[rgb]{0.94,0.16,0.16}{#1}}
\newcommand{\ErrorTok}[1]{\textcolor[rgb]{0.64,0.00,0.00}{\textbf{#1}}}
\newcommand{\NormalTok}[1]{#1}
\usepackage{longtable,booktabs}
\usepackage{graphicx,grffile}
\makeatletter
\def\maxwidth{\ifdim\Gin@nat@width>\linewidth\linewidth\else\Gin@nat@width\fi}
\def\maxheight{\ifdim\Gin@nat@height>\textheight\textheight\else\Gin@nat@height\fi}
\makeatother
% Scale images if necessary, so that they will not overflow the page
% margins by default, and it is still possible to overwrite the defaults
% using explicit options in \includegraphics[width, height, ...]{}
\setkeys{Gin}{width=\maxwidth,height=\maxheight,keepaspectratio}
\IfFileExists{parskip.sty}{%
\usepackage{parskip}
}{% else
\setlength{\parindent}{0pt}
\setlength{\parskip}{6pt plus 2pt minus 1pt}
}
\setlength{\emergencystretch}{3em}  % prevent overfull lines
\providecommand{\tightlist}{%
  \setlength{\itemsep}{0pt}\setlength{\parskip}{0pt}}
\setcounter{secnumdepth}{0}
% Redefines (sub)paragraphs to behave more like sections
\ifx\paragraph\undefined\else
\let\oldparagraph\paragraph
\renewcommand{\paragraph}[1]{\oldparagraph{#1}\mbox{}}
\fi
\ifx\subparagraph\undefined\else
\let\oldsubparagraph\subparagraph
\renewcommand{\subparagraph}[1]{\oldsubparagraph{#1}\mbox{}}
\fi

%%% Use protect on footnotes to avoid problems with footnotes in titles
\let\rmarkdownfootnote\footnote%
\def\footnote{\protect\rmarkdownfootnote}

%%% Change title format to be more compact
\usepackage{titling}

% Create subtitle command for use in maketitle
\providecommand{\subtitle}[1]{
  \posttitle{
    \begin{center}\large#1\end{center}
    }
}

\setlength{\droptitle}{-2em}

  \title{Introduction to RMarkdown}
    \pretitle{\vspace{\droptitle}\centering\huge}
  \posttitle{\par}
    \author{Abhinav Anand, IIMB}
    \preauthor{\centering\large\emph}
  \postauthor{\par}
      \predate{\centering\large\emph}
  \postdate{\par}
    \date{2019/06/18}

\linespread{1.25}

\begin{document}
\maketitle

\section{Setup}\label{setup}

The packages \texttt{rmarkdown} and \texttt{knitr} need to be installed
prior to running the commands below. To install, type in the console
\texttt{install.packages(c("knitr",\ "rmarkdown"))}.

\section{Background}\label{background}

A text processor such as MS Word uses a ``What-You-See-Is-What-You-Get''
(WYSIWYG) editor. We see the output of the ``rendered'' text directly.
However, to exercise finer control over the rendering, a ``markup''
language is better. Latex and HTML are examples of markup languages:
they contain ordinary text, as well as special commands that govern the
final rendering of the text. Markdown is an especially convenient markup
language that preserves the finer aspects of text formatting without
being too hard to read.

There are many programs for rendering documents written in Markdown into
documents in the .html, .pdf and .docx formats (among many others). R
Markdown extends Markdown to incorporate text formatting, mathematics,
figures, tables; as well as R code and its output directly into the
rendered document.

Behind the scenes, when we ``knit'' the file, R Markdown sends the
\texttt{.rmd} file to \texttt{knitr()} which executes all code chunks
and creates a new markdown file (extension \texttt{.md}) which includes
both the code and its output. This file is then used by \texttt{pandoc}
(essentially a free and open source document converter) to convert to
the desired format. This two-step workflow can help to create a very
wide range of formatting options for eventual publishing.

\subsection{Literate Programming and
Reproducibility}\label{literate-programming-and-reproducibility}

The objective for which RMarkdown proves most useful is its potential
for literate programming---a programming concept due to the famous
computer scientist Donald Knuth (also the father of TeX)---in which the
code and its explanation in a natural language occur in the same
document.

RMarkdown provides us with a way of including both code snippets and
marked up text to \emph{build} a document that contains codes, its
output as well as documentation. This also helps make research both
\emph{replicable} and \emph{reproducible} since the \texttt{.rmd} file
with the code can be sourced by readers themselves. The study can be
replicated with new data using the codes in the \texttt{.rmd} file while
the study can be reproduced by simply re-sourcing the document.
Reproducibility is also essential for our future selves who may need to
revisit old research projects.

\section{The Main Components of
RMarkdown}\label{the-main-components-of-rmarkdown}

Each RMarkdown document has three main components---header, text; and
code chunks.

\subsection{The (YAML) Header}\label{the-yaml-header}

The current file's header includes the following lines:

\begin{Shaded}
\begin{Highlighting}[]
\OperatorTok{---}
\NormalTok{title}\OperatorTok{:}\StringTok{ "Introduction to RMarkdown"}
\NormalTok{author}\OperatorTok{:}\StringTok{ }\NormalTok{Abhinav Anand}

\NormalTok{output}\OperatorTok{:}
\StringTok{  }\NormalTok{pdf_document}\OperatorTok{:}\StringTok{ }\CommentTok{# other options: .html, .docx etc.}
\StringTok{    }\NormalTok{keep_tex}\OperatorTok{:}\StringTok{ }\NormalTok{true }\CommentTok{# keeps the intermediate .tex document}

\NormalTok{fontsize}\OperatorTok{:}\StringTok{ }\NormalTok{11pt }\CommentTok{# latex parameter}
\NormalTok{documentclass}\OperatorTok{:}\StringTok{ }\NormalTok{article }\CommentTok{# latex class}
\NormalTok{geometry}\OperatorTok{:}\StringTok{ }\NormalTok{margin =}\StringTok{ }\FloatTok{1.}\NormalTok{5in }\CommentTok{# latex package}

\NormalTok{linkcolor}\OperatorTok{:}\StringTok{ }\NormalTok{blue }
\NormalTok{urlcolor}\OperatorTok{:}\StringTok{ }\NormalTok{red}
\NormalTok{citecolor}\OperatorTok{:}\StringTok{ }\NormalTok{magenta}

\NormalTok{citation_package}\OperatorTok{:}\StringTok{ }\NormalTok{natbib }\CommentTok{# latex style referencing}
\NormalTok{bibliography}\OperatorTok{:}\StringTok{ }\NormalTok{Working_Paper.bib}

\NormalTok{header}\OperatorTok{-}\NormalTok{includes}\OperatorTok{:}
\StringTok{   }\OperatorTok{-}\StringTok{ }\NormalTok{\textbackslash{}linespread\{}\FloatTok{1.25}\NormalTok{\}}


\OperatorTok{---}
\end{Highlighting}
\end{Shaded}

\subsection{Text}\label{text}

To insert a break between paragraphs, include a single completely blank
line.

For other text-formatting, see below:

\begin{enumerate}
\def\labelenumi{\arabic{enumi}.}
\tightlist
\item
  \texttt{*italic*}: \emph{italic}
\item
  \texttt{\_italic\_}: \emph{italic}
\item
  \texttt{**bold**}: \textbf{bold}
\item
  \texttt{\_\_bold\_\_}: \textbf{bold}
\item
  \texttt{superscript\^{}2\^{}}: superscript\textsuperscript{2}
\item
  \texttt{subscript\textasciitilde{}2\textasciitilde{}}:
  subscript\textsubscript{2}
\item
  ``Section'' heading: \texttt{\#}
\item
  ``Subsection'' heading: \texttt{\#\#}
\item
  ``Subsubsection'' heading: \texttt{\#\#\#}
\item
  Math: Just use the LaTeX style
\item
  Link:
  \texttt{\textless{}https://cran.r-project.org/.com\textgreater{}}:
  \url{https://cran.r-project.org/}
\item
  Word link: \texttt{{[}CRAN\ project{]}(https://cran.r-project.org/)}:
  \href{https://cran.r-project.org/}{CRAN project}
\item
  Simple Tables:
\end{enumerate}

\begin{Shaded}
\begin{Highlighting}[]
\NormalTok{First Column  }\OperatorTok{|}\StringTok{ }\NormalTok{Second Column}
\OperatorTok{-------------}\StringTok{ }\ErrorTok{|}\StringTok{ }\OperatorTok{-------------}
\ErrorTok{$}\NormalTok{c_}\DecValTok{1}\OperatorTok{$}\StringTok{  }\ErrorTok{|}\StringTok{ }\ErrorTok{$}\NormalTok{c_}\DecValTok{2}\OperatorTok{$}
\ErrorTok{$}\NormalTok{c_}\DecValTok{3}\OperatorTok{$}\StringTok{  }\ErrorTok{|}\StringTok{ }\ErrorTok{$}\NormalTok{c_}\DecValTok{4}\OperatorTok{$}
\end{Highlighting}
\end{Shaded}

\begin{longtable}[]{@{}ll@{}}
\toprule
First Column & Second Column\tabularnewline
\midrule
\endhead
\(c_1\) & \(c_2\)\tabularnewline
\(c_3\) & \(c_4\)\tabularnewline
\bottomrule
\end{longtable}

\begin{enumerate}
\def\labelenumi{\arabic{enumi}.}
\setcounter{enumi}{12}
\tightlist
\item
  Lists:
\end{enumerate}

\begin{Shaded}
\begin{Highlighting}[]
\OperatorTok{*}\StringTok{   }\NormalTok{Bulleted list item }\DecValTok{1}

\OperatorTok{*}\StringTok{   }\NormalTok{Item }\DecValTok{2}

    \OperatorTok{*}\StringTok{ }\NormalTok{Item 2a}

    \OperatorTok{*}\StringTok{ }\NormalTok{Item 2b}

\DecValTok{1}\NormalTok{.  Numbered list item }\DecValTok{1}
\DecValTok{2}\NormalTok{. Item }\DecValTok{2}
\DecValTok{3}\NormalTok{. Item }\DecValTok{3}
\end{Highlighting}
\end{Shaded}

\begin{itemize}
\item
  Bulleted list item 1
\item
  Item 2

  \begin{itemize}
  \item
    Item 2a
  \item
    Item 2b
  \end{itemize}
\end{itemize}

\begin{enumerate}
\def\labelenumi{\arabic{enumi}.}
\tightlist
\item
  Numbered list item 1
\item
  Item 2
\item
  Item 3
\end{enumerate}

\subsubsection{Table}\label{table}

If additional table formatting is needed, use the \texttt{knitr::kable}
function.

\begin{Shaded}
\begin{Highlighting}[]
\NormalTok{knitr}\OperatorTok{::}\KeywordTok{kable}\NormalTok{(gapminder}\OperatorTok{::}\NormalTok{gapminder[}\DecValTok{1}\OperatorTok{:}\DecValTok{10}\NormalTok{, ],}
             \DataTypeTok{caption =} \StringTok{"The first 10 rows of gapminder"}\NormalTok{)}
\end{Highlighting}
\end{Shaded}

\begin{longtable}[]{@{}llrrrr@{}}
\caption{The first 10 rows of gapminder}\tabularnewline
\toprule
country & continent & year & lifeExp & pop & gdpPercap\tabularnewline
\midrule
\endfirsthead
\toprule
country & continent & year & lifeExp & pop & gdpPercap\tabularnewline
\midrule
\endhead
Afghanistan & Asia & 1952 & 28.801 & 8425333 & 779.4453\tabularnewline
Afghanistan & Asia & 1957 & 30.332 & 9240934 & 820.8530\tabularnewline
Afghanistan & Asia & 1962 & 31.997 & 10267083 & 853.1007\tabularnewline
Afghanistan & Asia & 1967 & 34.020 & 11537966 & 836.1971\tabularnewline
Afghanistan & Asia & 1972 & 36.088 & 13079460 & 739.9811\tabularnewline
Afghanistan & Asia & 1977 & 38.438 & 14880372 & 786.1134\tabularnewline
Afghanistan & Asia & 1982 & 39.854 & 12881816 & 978.0114\tabularnewline
Afghanistan & Asia & 1987 & 40.822 & 13867957 & 852.3959\tabularnewline
Afghanistan & Asia & 1992 & 41.674 & 16317921 & 649.3414\tabularnewline
Afghanistan & Asia & 1997 & 41.763 & 22227415 & 635.3414\tabularnewline
\bottomrule
\end{longtable}

\subsubsection{Bibliography and
Citations}\label{bibliography-and-citations}

Declare in YAML header as shown above. To cite, pre-fix \texttt{@} with
the citation key in the bibliography file.

For example:

\begin{enumerate}
\def\labelenumi{\arabic{enumi}.}
\tightlist
\item
  This is the simple style of citation: \texttt{@FH:2009} yielding
  Foster and Hart (2009)
\item
  For multiple citations, a semi-colon is used:
  \texttt{@Rachev:2008;\ @Fama:1963}---Rachev, Stoyanov, and Fabozzi
  (2008); Fama (1963)
\item
  For citing within parentheses, use: square brackets
  \texttt{{[}@Bollerslev:1986{]}}, (Bollerslev 1986)
\end{enumerate}

It is common practice to end the rmarkdown file with a section header
for the bibliography, such as \texttt{\#\ References} or
\texttt{\#\ Bibliography} since unlike Latex, there is no automatic
section heading for the bibliography.

\subsection{Code Chunks}\label{code-chunks}

All inline codes must be within backticks along with the symbol
\texttt{r}.

For more involved computation, we need to use code chunks, enclosed
within three backticks, followed by \texttt{r} and then followed by
three closing backticks:
\texttt{\textasciigrave{}\textasciigrave{}\textasciigrave{}\{r\ \}}
followed by
\texttt{\textasciigrave{}\textasciigrave{}\textasciigrave{}}. It is
considered good practice to include a short name following the symbol
\texttt{r}.

\subsubsection{Chunk options}\label{chunk-options}

The following are the main options:

\begin{enumerate}
\def\labelenumi{\arabic{enumi}.}
\tightlist
\item
  \texttt{eval\ =\ FALSE}: This will ensure no evaluation of the code
  chunk.
\item
  \texttt{include\ =\ FALSE}: Run the code but show neither the code nor
  its results.
\item
  \texttt{echo\ =\ FALSE}: Run the code, show the results but don't show
  the code.
\item
  \texttt{message\ =\ FALSE} or \texttt{warning\ =\ FALSE}: Don't show
  messages or warnings.
\end{enumerate}

If these choices are global---the same for all code chunks in the
document---then we can include the options at the beginning of the
document via:

\begin{Shaded}
\begin{Highlighting}[]
\NormalTok{knitr}\OperatorTok{::}\NormalTok{opts_chunk}\OperatorTok{$}\KeywordTok{set}\NormalTok{(}\DataTypeTok{echo =}\NormalTok{ T, }
                      \DataTypeTok{warning =}\NormalTok{ T, }
                      \DataTypeTok{message =}\NormalTok{ F, }
                      \DataTypeTok{eval =}\NormalTok{ T, }
                      \DataTypeTok{include =}\NormalTok{ T}
\NormalTok{                      )}
\end{Highlighting}
\end{Shaded}

If we need to override global options in a code chunk, we may simply set
that parameter manually after declaring the code chunk name: say,
\texttt{eval\ =\ F} for one particular chunk.

\section*{References}\label{references}
\addcontentsline{toc}{section}{References}

\hypertarget{refs}{}
\hypertarget{ref-Bollerslev:1986}{}
Bollerslev, T. 1986. ``Generalized Autoregressive Conditional
Heteroskedasticity.'' \emph{Journal of Econometrics} 31: 307--27.

\hypertarget{ref-Fama:1963}{}
Fama, Eugene F. 1963. ``Mandelbrot and the Stable Paretian Hypothesis.''
\emph{Journal of Business} 36: 420--29.

\hypertarget{ref-FH:2009}{}
Foster, Dean P., and Sergiu Hart. 2009. ``An Operational Measure of
Riskiness.'' \emph{Journal of Political Economy} 117 (5): 785--814.

\hypertarget{ref-Rachev:2008}{}
Rachev, Svetlozar T., Stoyan V. Stoyanov, and Frank J. Fabozzi. 2008.
\emph{Advanced Stochastic Models, Risk Assessment, and Portfolio
Optimization}. Hoboken, New Jersey: John Wiley; Sons.


\end{document}
