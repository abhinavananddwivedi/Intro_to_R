\documentclass[12pt,]{article}
\usepackage{lmodern}
\usepackage{amssymb,amsmath}
\usepackage{ifxetex,ifluatex}
\usepackage{fixltx2e} % provides \textsubscript
\ifnum 0\ifxetex 1\fi\ifluatex 1\fi=0 % if pdftex
  \usepackage[T1]{fontenc}
  \usepackage[utf8]{inputenc}
\else % if luatex or xelatex
  \ifxetex
    \usepackage{mathspec}
  \else
    \usepackage{fontspec}
  \fi
  \defaultfontfeatures{Ligatures=TeX,Scale=MatchLowercase}
\fi
% use upquote if available, for straight quotes in verbatim environments
\IfFileExists{upquote.sty}{\usepackage{upquote}}{}
% use microtype if available
\IfFileExists{microtype.sty}{%
\usepackage{microtype}
\UseMicrotypeSet[protrusion]{basicmath} % disable protrusion for tt fonts
}{}
\usepackage[margin = 1.5in]{geometry}
\usepackage{hyperref}
\PassOptionsToPackage{usenames,dvipsnames}{color} % color is loaded by hyperref
\hypersetup{unicode=true,
            pdftitle={More Tips and Tricks},
            pdfauthor={Abhinav Anand, IIMB},
            colorlinks=true,
            linkcolor=blue,
            citecolor=magenta,
            urlcolor=red,
            breaklinks=true}
\urlstyle{same}  % don't use monospace font for urls
\usepackage{color}
\usepackage{fancyvrb}
\newcommand{\VerbBar}{|}
\newcommand{\VERB}{\Verb[commandchars=\\\{\}]}
\DefineVerbatimEnvironment{Highlighting}{Verbatim}{commandchars=\\\{\}}
% Add ',fontsize=\small' for more characters per line
\usepackage{framed}
\definecolor{shadecolor}{RGB}{248,248,248}
\newenvironment{Shaded}{\begin{snugshade}}{\end{snugshade}}
\newcommand{\KeywordTok}[1]{\textcolor[rgb]{0.13,0.29,0.53}{\textbf{#1}}}
\newcommand{\DataTypeTok}[1]{\textcolor[rgb]{0.13,0.29,0.53}{#1}}
\newcommand{\DecValTok}[1]{\textcolor[rgb]{0.00,0.00,0.81}{#1}}
\newcommand{\BaseNTok}[1]{\textcolor[rgb]{0.00,0.00,0.81}{#1}}
\newcommand{\FloatTok}[1]{\textcolor[rgb]{0.00,0.00,0.81}{#1}}
\newcommand{\ConstantTok}[1]{\textcolor[rgb]{0.00,0.00,0.00}{#1}}
\newcommand{\CharTok}[1]{\textcolor[rgb]{0.31,0.60,0.02}{#1}}
\newcommand{\SpecialCharTok}[1]{\textcolor[rgb]{0.00,0.00,0.00}{#1}}
\newcommand{\StringTok}[1]{\textcolor[rgb]{0.31,0.60,0.02}{#1}}
\newcommand{\VerbatimStringTok}[1]{\textcolor[rgb]{0.31,0.60,0.02}{#1}}
\newcommand{\SpecialStringTok}[1]{\textcolor[rgb]{0.31,0.60,0.02}{#1}}
\newcommand{\ImportTok}[1]{#1}
\newcommand{\CommentTok}[1]{\textcolor[rgb]{0.56,0.35,0.01}{\textit{#1}}}
\newcommand{\DocumentationTok}[1]{\textcolor[rgb]{0.56,0.35,0.01}{\textbf{\textit{#1}}}}
\newcommand{\AnnotationTok}[1]{\textcolor[rgb]{0.56,0.35,0.01}{\textbf{\textit{#1}}}}
\newcommand{\CommentVarTok}[1]{\textcolor[rgb]{0.56,0.35,0.01}{\textbf{\textit{#1}}}}
\newcommand{\OtherTok}[1]{\textcolor[rgb]{0.56,0.35,0.01}{#1}}
\newcommand{\FunctionTok}[1]{\textcolor[rgb]{0.00,0.00,0.00}{#1}}
\newcommand{\VariableTok}[1]{\textcolor[rgb]{0.00,0.00,0.00}{#1}}
\newcommand{\ControlFlowTok}[1]{\textcolor[rgb]{0.13,0.29,0.53}{\textbf{#1}}}
\newcommand{\OperatorTok}[1]{\textcolor[rgb]{0.81,0.36,0.00}{\textbf{#1}}}
\newcommand{\BuiltInTok}[1]{#1}
\newcommand{\ExtensionTok}[1]{#1}
\newcommand{\PreprocessorTok}[1]{\textcolor[rgb]{0.56,0.35,0.01}{\textit{#1}}}
\newcommand{\AttributeTok}[1]{\textcolor[rgb]{0.77,0.63,0.00}{#1}}
\newcommand{\RegionMarkerTok}[1]{#1}
\newcommand{\InformationTok}[1]{\textcolor[rgb]{0.56,0.35,0.01}{\textbf{\textit{#1}}}}
\newcommand{\WarningTok}[1]{\textcolor[rgb]{0.56,0.35,0.01}{\textbf{\textit{#1}}}}
\newcommand{\AlertTok}[1]{\textcolor[rgb]{0.94,0.16,0.16}{#1}}
\newcommand{\ErrorTok}[1]{\textcolor[rgb]{0.64,0.00,0.00}{\textbf{#1}}}
\newcommand{\NormalTok}[1]{#1}
\usepackage{graphicx,grffile}
\makeatletter
\def\maxwidth{\ifdim\Gin@nat@width>\linewidth\linewidth\else\Gin@nat@width\fi}
\def\maxheight{\ifdim\Gin@nat@height>\textheight\textheight\else\Gin@nat@height\fi}
\makeatother
% Scale images if necessary, so that they will not overflow the page
% margins by default, and it is still possible to overwrite the defaults
% using explicit options in \includegraphics[width, height, ...]{}
\setkeys{Gin}{width=\maxwidth,height=\maxheight,keepaspectratio}
\IfFileExists{parskip.sty}{%
\usepackage{parskip}
}{% else
\setlength{\parindent}{0pt}
\setlength{\parskip}{6pt plus 2pt minus 1pt}
}
\setlength{\emergencystretch}{3em}  % prevent overfull lines
\providecommand{\tightlist}{%
  \setlength{\itemsep}{0pt}\setlength{\parskip}{0pt}}
\setcounter{secnumdepth}{0}
% Redefines (sub)paragraphs to behave more like sections
\ifx\paragraph\undefined\else
\let\oldparagraph\paragraph
\renewcommand{\paragraph}[1]{\oldparagraph{#1}\mbox{}}
\fi
\ifx\subparagraph\undefined\else
\let\oldsubparagraph\subparagraph
\renewcommand{\subparagraph}[1]{\oldsubparagraph{#1}\mbox{}}
\fi

%%% Use protect on footnotes to avoid problems with footnotes in titles
\let\rmarkdownfootnote\footnote%
\def\footnote{\protect\rmarkdownfootnote}

%%% Change title format to be more compact
\usepackage{titling}

% Create subtitle command for use in maketitle
\providecommand{\subtitle}[1]{
  \posttitle{
    \begin{center}\large#1\end{center}
    }
}

\setlength{\droptitle}{-2em}

  \title{More Tips and Tricks}
    \pretitle{\vspace{\droptitle}\centering\huge}
  \posttitle{\par}
    \author{Abhinav Anand, IIMB}
    \preauthor{\centering\large\emph}
  \postauthor{\par}
      \predate{\centering\large\emph}
  \postdate{\par}
    \date{2019/06/18}

\linespread{1.3}

\begin{document}
\maketitle

\section{Scripts in R}\label{scripts-in-r}

Scripts are files (ending in \texttt{.R}) that contain sequences of
instructions that we can command a machine to follow. Consier a typical
empirical project: we need to read data stored in some type of file;
then clean and tidy it, then process it and use descriptive statistics
and plots to delineate its features etc. All this can be achieved by
storing a sequence of instructions in a script file. Data can be read
using the the \texttt{read\_csv()} function, processing can be done
using the package \texttt{dplyr} etc.

\section{Linear regression in R}\label{linear-regression-in-r}

Generally a linear model takes the following form:

\[
y = \beta_0 + \beta_1x_1 + \hdots + \beta_mx_m + u
\] where \(u_{n\times 1}\) is the error term. This setup corresponds to
an overdetermined linear system of equations leading to a least squares
solution:

\[
\hat{\beta} = (X^{\top} X)^{-1} X^{\top}y 
\]

where the explanatory matrix \(X_{m\times n}\) contains independent
variables \(x_1,\hdots,x_m\) as column vectors of size \(n\times 1\).

One of the strengths of R is the flexibility and support it offers for
linear regression modeling. In order to illustrate it more fully, let us
consider data for India in the \texttt{gapminder} dataset.

\begin{Shaded}
\begin{Highlighting}[]
\NormalTok{data_Ind <-}\StringTok{ }\NormalTok{gapminder}\OperatorTok{::}\NormalTok{gapminder }\OperatorTok\StringTok{ }
\StringTok{  }\NormalTok{dplyr}\OperatorTok{::}\KeywordTok{filter}\NormalTok{(country }\OperatorTok{==}\StringTok{ "India"}\NormalTok{)}

\KeywordTok{ggplot}\NormalTok{(data_Ind, }\KeywordTok{aes}\NormalTok{(year, gdpPercap)) }\OperatorTok{+}
\StringTok{  }\KeywordTok{geom_point}\NormalTok{() }\OperatorTok{+}
\StringTok{  }\KeywordTok{geom_line}\NormalTok{() }
\end{Highlighting}
\end{Shaded}

\begin{center}\includegraphics{Intro_tips_tricks_files/figure-latex/data_Ind_GDP-1} \end{center}

We see that there has been a large increase in GDP per capita in India.
A similar trend is observed for life expectancy:

\begin{Shaded}
\begin{Highlighting}[]
\KeywordTok{ggplot}\NormalTok{(data_Ind, }\KeywordTok{aes}\NormalTok{(year, lifeExp)) }\OperatorTok{+}
\StringTok{  }\KeywordTok{geom_point}\NormalTok{() }\OperatorTok{+}
\StringTok{  }\KeywordTok{geom_line}\NormalTok{() }
\end{Highlighting}
\end{Shaded}

\begin{center}\includegraphics{Intro_tips_tricks_files/figure-latex/data_Ind_lifeexp-1} \end{center}

What about the relationship between the two? For example, (all else
equal) does GDP per capita of India explain the life expectancy trends
observed?

\begin{Shaded}
\begin{Highlighting}[]
\KeywordTok{ggplot}\NormalTok{(data_Ind, }\KeywordTok{aes}\NormalTok{(gdpPercap, lifeExp)) }\OperatorTok{+}
\StringTok{  }\KeywordTok{geom_point}\NormalTok{() }
\end{Highlighting}
\end{Shaded}

\begin{center}\includegraphics{Intro_tips_tricks_files/figure-latex/data_Ind_scatter-1} \end{center}

This suggests that the two variables share a positive relation. We can
try to check this by means of a linear regression in the following way:

\[
\text{life exp} = \beta_0 + \beta_1 \text{(gdp percap)} + u
\]

In order to implement this step in R is via the following:

\begin{Shaded}
\begin{Highlighting}[]
\NormalTok{lm_formula <-}\StringTok{ }\NormalTok{lifeExp }\OperatorTok{~}\StringTok{ }\NormalTok{gdpPercap}

\NormalTok{lm_life_gdppc <-}\StringTok{ }\KeywordTok{lm}\NormalTok{(}\DataTypeTok{data =}\NormalTok{ data_Ind, }\DataTypeTok{formula =}\NormalTok{ lm_formula)}

\KeywordTok{summary}\NormalTok{(lm_life_gdppc)}
\end{Highlighting}
\end{Shaded}

\begin{verbatim}
## 
## Call:
## lm(formula = lm_formula, data = data_Ind)
## 
## Residuals:
##    Min     1Q Median     3Q    Max 
## -9.155 -4.931  1.284  4.576  6.437 
## 
## Coefficients:
##              Estimate Std. Error t value Pr(>|t|)    
## (Intercept) 39.423336   3.659432  10.773    8e-07 ***
## gdpPercap    0.012998   0.003075   4.227  0.00175 ** 
## ---
## Signif. codes:  0 '***' 0.001 '**' 0.01 '*' 0.05 '.' 0.1 ' ' 1
## 
## Residual standard error: 5.816 on 10 degrees of freedom
## Multiple R-squared:  0.6411, Adjusted R-squared:  0.6052 
## F-statistic: 17.86 on 1 and 10 DF,  p-value: 0.001753
\end{verbatim}

\begin{Shaded}
\begin{Highlighting}[]
\KeywordTok{ggplot}\NormalTok{(data_Ind, }\KeywordTok{aes}\NormalTok{(gdpPercap, lifeExp)) }\OperatorTok{+}
\StringTok{  }\KeywordTok{geom_point}\NormalTok{() }\OperatorTok{+}
\StringTok{  }\KeywordTok{geom_smooth}\NormalTok{(}\DataTypeTok{method =} \StringTok{"lm"}\NormalTok{)}
\end{Highlighting}
\end{Shaded}

\begin{center}\includegraphics{Intro_tips_tricks_files/figure-latex/data_Ind_lm-1} \end{center}

What is this object \texttt{lm\_life\_gdppc}? What is its structure? We
can quickly check by accessing its contents:

\begin{Shaded}
\begin{Highlighting}[]
\KeywordTok{names}\NormalTok{(lm_life_gdppc)}
\end{Highlighting}
\end{Shaded}

\begin{verbatim}
##  [1] "coefficients"  "residuals"     "effects"       "rank"         
##  [5] "fitted.values" "assign"        "qr"            "df.residual"  
##  [9] "xlevels"       "call"          "terms"         "model"
\end{verbatim}

What about some subset of data, say the period before 1990?

\begin{Shaded}
\begin{Highlighting}[]
\KeywordTok{lm}\NormalTok{(}\KeywordTok{filter}\NormalTok{(data_Ind, year }\OperatorTok{<=}\StringTok{ }\DecValTok{1990}\NormalTok{), }\DataTypeTok{formula =}\NormalTok{ lm_formula) }\OperatorTok\StringTok{ }
\StringTok{  }\KeywordTok{summary}\NormalTok{()}
\end{Highlighting}
\end{Shaded}

\begin{verbatim}
## 
## Call:
## lm(formula = lm_formula, data = filter(data_Ind, year <= 1990))
## 
## Residuals:
##     Min      1Q  Median      3Q     Max 
## -2.8626 -1.0747 -0.1943  1.4541  2.5804 
## 
## Coefficients:
##             Estimate Std. Error t value Pr(>|t|)    
## (Intercept)  9.80149    3.83775   2.554   0.0433 *  
## gdpPercap    0.05286    0.00515  10.264 4.99e-05 ***
## ---
## Signif. codes:  0 '***' 0.001 '**' 0.01 '*' 0.05 '.' 0.1 ' ' 1
## 
## Residual standard error: 1.945 on 6 degrees of freedom
## Multiple R-squared:  0.9461, Adjusted R-squared:  0.9371 
## F-statistic: 105.3 on 1 and 6 DF,  p-value: 4.992e-05
\end{verbatim}

\begin{Shaded}
\begin{Highlighting}[]
\KeywordTok{ggplot}\NormalTok{(}\KeywordTok{filter}\NormalTok{(data_Ind, year }\OperatorTok{<=}\StringTok{ }\DecValTok{1990}\NormalTok{), }\KeywordTok{aes}\NormalTok{(lifeExp, gdpPercap)) }\OperatorTok{+}
\StringTok{  }\KeywordTok{geom_point}\NormalTok{() }\OperatorTok{+}
\StringTok{  }\KeywordTok{geom_smooth}\NormalTok{(}\DataTypeTok{method =} \StringTok{"lm"}\NormalTok{)}
\end{Highlighting}
\end{Shaded}

\begin{center}\includegraphics{Intro_tips_tricks_files/figure-latex/data_Ind_lm_subset-1} \end{center}

\subsection{Nonlinear relationships}\label{nonlinear-relationships}

The plot between the dependent and independent variable suggest a
nonlinear relationship. Can we test this simply? Let's consider the
following modification:

\[
\text{life exp} = \beta_0 + \beta_1 \text{(gdp percap)}^2 + u
\]

In general, R can accommodate independent variables involving
mathematical operators in a regression equation with the function
\texttt{I()}.

\begin{Shaded}
\begin{Highlighting}[]
\NormalTok{lm_formula_quad <-}\StringTok{ }\NormalTok{lifeExp }\OperatorTok{~}\StringTok{ }\KeywordTok{I}\NormalTok{(gdpPercap)}\OperatorTok{^}\DecValTok{2}

\NormalTok{lm_life_gdppc_quad <-}\StringTok{ }\KeywordTok{lm}\NormalTok{(}\DataTypeTok{data =}\NormalTok{ data_Ind, }\DataTypeTok{formula =}\NormalTok{ lm_formula_quad)}

\KeywordTok{summary}\NormalTok{(lm_life_gdppc_quad)}
\end{Highlighting}
\end{Shaded}

\begin{verbatim}
## 
## Call:
## lm(formula = lm_formula_quad, data = data_Ind)
## 
## Residuals:
##    Min     1Q Median     3Q    Max 
## -9.155 -4.931  1.284  4.576  6.437 
## 
## Coefficients:
##               Estimate Std. Error t value Pr(>|t|)    
## (Intercept)  39.423336   3.659432  10.773    8e-07 ***
## I(gdpPercap)  0.012998   0.003075   4.227  0.00175 ** 
## ---
## Signif. codes:  0 '***' 0.001 '**' 0.01 '*' 0.05 '.' 0.1 ' ' 1
## 
## Residual standard error: 5.816 on 10 degrees of freedom
## Multiple R-squared:  0.6411, Adjusted R-squared:  0.6052 
## F-statistic: 17.86 on 1 and 10 DF,  p-value: 0.001753
\end{verbatim}

\begin{Shaded}
\begin{Highlighting}[]
\KeywordTok{ggplot}\NormalTok{(data_Ind, }\KeywordTok{aes}\NormalTok{(}\DataTypeTok{x =}\NormalTok{ gdpPercap, }\DataTypeTok{y =}\NormalTok{ lifeExp)) }\OperatorTok{+}
\StringTok{  }\KeywordTok{geom_point}\NormalTok{() }\OperatorTok{+}
\StringTok{  }\KeywordTok{stat_smooth}\NormalTok{(}\DataTypeTok{method =} \StringTok{"lm"}\NormalTok{, }
              \DataTypeTok{formula =}\NormalTok{ y }\OperatorTok{~}\StringTok{ }\KeywordTok{poly}\NormalTok{(x, }\DecValTok{2}\NormalTok{), }\CommentTok{#polynomial order 2}
              \DataTypeTok{size =} \FloatTok{0.8}\NormalTok{,}
              \DataTypeTok{linetype =} \StringTok{"dashed"}
\NormalTok{              )}
\end{Highlighting}
\end{Shaded}

\begin{center}\includegraphics{Intro_tips_tricks_files/figure-latex/data_Ind_lm_quad-1} \end{center}

\begin{Shaded}
\begin{Highlighting}[]
\CommentTok{# What about higher order polynomials?}
\KeywordTok{ggplot}\NormalTok{(data_Ind, }\KeywordTok{aes}\NormalTok{(}\DataTypeTok{x =}\NormalTok{ gdpPercap, }\DataTypeTok{y =}\NormalTok{ lifeExp)) }\OperatorTok{+}
\StringTok{  }\KeywordTok{geom_point}\NormalTok{() }\OperatorTok{+}
\StringTok{  }\KeywordTok{stat_smooth}\NormalTok{(}\DataTypeTok{method =} \StringTok{"lm"}\NormalTok{, }
              \DataTypeTok{formula =}\NormalTok{ y }\OperatorTok{~}\StringTok{ }\KeywordTok{poly}\NormalTok{(x, }\DecValTok{3}\NormalTok{), }\CommentTok{#polynomial order 3}
              \DataTypeTok{size =} \FloatTok{0.8}\NormalTok{,}
              \DataTypeTok{linetype =} \StringTok{"dotdash"}
\NormalTok{              )}
\end{Highlighting}
\end{Shaded}

\begin{center}\includegraphics{Intro_tips_tricks_files/figure-latex/data_Ind_lm_quad-2} \end{center}

Are visually better fits also evidence of better underlying models? This
is a hard question to answer in general---all else equal we prefer
models that are parsimonious (have fewer explanatory variables).

\section{Functional Programming in R}\label{functional-programming-in-r}

Another very powerful feature of R is its support for functional
programming, which in general, involves applying functions to arrays,
dataframes, lists etc.

For example, how should one compute the mean across rows of a matrix?

\begin{Shaded}
\begin{Highlighting}[]
\NormalTok{df <-}\StringTok{ }\KeywordTok{data.frame}\NormalTok{(}\DataTypeTok{C_1 =} \KeywordTok{rnorm}\NormalTok{(}\DecValTok{10}\NormalTok{, }\DecValTok{0}\NormalTok{, }\DecValTok{1}\NormalTok{), }
                 \DataTypeTok{C_2 =} \KeywordTok{rnorm}\NormalTok{(}\DecValTok{10}\NormalTok{, }\DecValTok{1}\NormalTok{, }\DecValTok{2}\NormalTok{),}
                 \DataTypeTok{C_3 =} \KeywordTok{rnorm}\NormalTok{(}\DecValTok{10}\NormalTok{, }\DecValTok{2}\NormalTok{, }\DecValTok{3}\NormalTok{)}
\NormalTok{                 )}

\KeywordTok{head}\NormalTok{(df)}
\end{Highlighting}
\end{Shaded}

\begin{verbatim}
##          C_1        C_2         C_3
## 1  0.5381310  2.0581087 -0.87550785
## 2  1.6755015  1.3252460  0.76989995
## 3 -1.2965104  5.2609430  3.65960670
## 4  0.4062685  0.2999016  4.13398703
## 5  1.0145668  3.8479344 -3.90830778
## 6 -1.0538232 -0.2574456 -0.07364864
\end{verbatim}

\begin{Shaded}
\begin{Highlighting}[]
\CommentTok{# One way to solve the problem}
\NormalTok{rmean_}\DecValTok{1}\NormalTok{ <-}\StringTok{ }\KeywordTok{rowMeans}\NormalTok{(df)}
\KeywordTok{print}\NormalTok{(rmean_}\DecValTok{1}\NormalTok{)}
\end{Highlighting}
\end{Shaded}

\begin{verbatim}
##  [1]  0.5735773  1.2568825  2.5413464  1.6133857  0.3180645 -0.4616391
##  [7]  0.2895387  0.3109647  1.6144940  0.8859763
\end{verbatim}

\begin{Shaded}
\begin{Highlighting}[]
\CommentTok{# Another more 'functional' way}
\NormalTok{func_mean <-}\StringTok{ }\ControlFlowTok{function}\NormalTok{(vec)}
\NormalTok{\{}
  \KeywordTok{return}\NormalTok{(}\KeywordTok{mean}\NormalTok{(vec, }\DataTypeTok{na.rm =}\NormalTok{ T))}
\NormalTok{\}}

\CommentTok{# Apply function on rows}
\NormalTok{rmean_}\DecValTok{2}\NormalTok{ <-}\StringTok{ }\KeywordTok{apply}\NormalTok{(df, }\DecValTok{1}\NormalTok{, func_mean) }
\KeywordTok{print}\NormalTok{(rmean_}\DecValTok{1}\NormalTok{)}
\end{Highlighting}
\end{Shaded}

\begin{verbatim}
##  [1]  0.5735773  1.2568825  2.5413464  1.6133857  0.3180645 -0.4616391
##  [7]  0.2895387  0.3109647  1.6144940  0.8859763
\end{verbatim}

\begin{Shaded}
\begin{Highlighting}[]
\CommentTok{# Apply function on columns}
\NormalTok{rmean_}\DecValTok{3}\NormalTok{ <-}\StringTok{ }\KeywordTok{apply}\NormalTok{(df, }\DecValTok{2}\NormalTok{, func_mean)}
\KeywordTok{print}\NormalTok{(rmean_}\DecValTok{3}\NormalTok{)}
\end{Highlighting}
\end{Shaded}

\begin{verbatim}
##          C_1          C_2          C_3 
## -0.002382187  1.569520037  1.115639449
\end{verbatim}

Note how to use the \texttt{apply()} function. We \texttt{apply()} the
function over rows or columns or other dimensions. In general that's the
philosophy of the \texttt{apply()} family of functions, which includes
functions \texttt{lapply()} (list-apply) and \texttt{sapply()}
(simplify-apply) etc. The function \texttt{lapply} returns a list and
\texttt{sapply} a vector (if possible). In both cases the first argument
is a list (or dataframe) and the second argument is the name of a
function.

What is a list? It's essentially a more general version of a dataframe
and can contain not only dissimilar data types but also, say dataframes
within them.

\begin{Shaded}
\begin{Highlighting}[]
\NormalTok{temp_list <-}\StringTok{ }\KeywordTok{list}\NormalTok{(}\DataTypeTok{a =} \KeywordTok{runif}\NormalTok{(}\DecValTok{10}\NormalTok{),}
                  \DataTypeTok{b =} \StringTok{"Happy birthday"}\NormalTok{,}
                  \DataTypeTok{c =} \KeywordTok{data.frame}\NormalTok{(}\DataTypeTok{x =} \KeywordTok{rnorm}\NormalTok{(}\DecValTok{10}\NormalTok{, }\DecValTok{0}\NormalTok{, }\DecValTok{1}\NormalTok{)),}
                  \DataTypeTok{d =} \KeywordTok{sample}\NormalTok{(letters, }\DecValTok{7}\NormalTok{, }\DataTypeTok{replace =} \OtherTok{TRUE}\NormalTok{)}
\NormalTok{                  )}
\KeywordTok{str}\NormalTok{(temp_list) }\CommentTok{#structure of the list}
\end{Highlighting}
\end{Shaded}

\begin{verbatim}
## List of 4
##  $ a: num [1:10] 0.704 0.597 0.584 0.173 0.474 ...
##  $ b: chr "Happy birthday"
##  $ c:'data.frame':   10 obs. of  1 variable:
##   ..$ x: num [1:10] 0.347 -2.513 -0.394 0.044 -1.187 ...
##  $ d: chr [1:7] "e" "u" "z" "j" ...
\end{verbatim}

\begin{Shaded}
\begin{Highlighting}[]
\CommentTok{# 1apply() is used to apply the same function to each}
\CommentTok{# "cell" of the list}
\KeywordTok{lapply}\NormalTok{(temp_list, is.numeric) }\CommentTok{#check if each cell is numeric}
\end{Highlighting}
\end{Shaded}

\begin{verbatim}
## $a
## [1] TRUE
## 
## $b
## [1] FALSE
## 
## $c
## [1] FALSE
## 
## $d
## [1] FALSE
\end{verbatim}

\begin{Shaded}
\begin{Highlighting}[]
\CommentTok{# contrast with sapply()}
\KeywordTok{sapply}\NormalTok{(temp_list, is.numeric)}
\end{Highlighting}
\end{Shaded}

\begin{verbatim}
##     a     b     c     d 
##  TRUE FALSE FALSE FALSE
\end{verbatim}

\subsection{\texorpdfstring{The \texttt{map()} family from
\texttt{purrr}}{The map() family from purrr}}\label{the-map-family-from-purrr}

The \texttt{map} function does the exact same operation as
\texttt{apply} but is consistent with the output format type. For
example \texttt{map()} returns a list, \texttt{map\_dbl()} returns a
double type vector, \texttt{map\_int()} returns an integer type vector
etc. As with \texttt{read\_csv}, the \texttt{map} family improves upon
the base R code by being faster and more consistent.

\begin{Shaded}
\begin{Highlighting}[]
\KeywordTok{map}\NormalTok{(df, mean)}
\end{Highlighting}
\end{Shaded}

\begin{verbatim}
## $C_1
## [1] -0.002382187
## 
## $C_2
## [1] 1.56952
## 
## $C_3
## [1] 1.115639
\end{verbatim}

\begin{Shaded}
\begin{Highlighting}[]
\KeywordTok{map_dbl}\NormalTok{(df, median)}
\end{Highlighting}
\end{Shaded}

\begin{verbatim}
##       C_1       C_2       C_3 
## 0.4158354 0.8125738 1.1505824
\end{verbatim}

\begin{Shaded}
\begin{Highlighting}[]
\CommentTok{# Also compare this}
\NormalTok{z <-}\StringTok{ }\KeywordTok{list}\NormalTok{(}\DataTypeTok{x =} \DecValTok{1}\OperatorTok{:}\DecValTok{3}\NormalTok{, }\DataTypeTok{y =} \DecValTok{4}\OperatorTok{:}\DecValTok{5}\NormalTok{)}
\KeywordTok{map_int}\NormalTok{(z, length) }\CommentTok{#names are preserved}
\end{Highlighting}
\end{Shaded}

\begin{verbatim}
## x y 
## 3 2
\end{verbatim}


\end{document}
