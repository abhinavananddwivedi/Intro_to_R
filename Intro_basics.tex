\documentclass[12pt,]{article}
\usepackage{lmodern}
\usepackage{amssymb,amsmath}
\usepackage{ifxetex,ifluatex}
\usepackage{fixltx2e} % provides \textsubscript
\ifnum 0\ifxetex 1\fi\ifluatex 1\fi=0 % if pdftex
  \usepackage[T1]{fontenc}
  \usepackage[utf8]{inputenc}
\else % if luatex or xelatex
  \ifxetex
    \usepackage{mathspec}
  \else
    \usepackage{fontspec}
  \fi
  \defaultfontfeatures{Ligatures=TeX,Scale=MatchLowercase}
\fi
% use upquote if available, for straight quotes in verbatim environments
\IfFileExists{upquote.sty}{\usepackage{upquote}}{}
% use microtype if available
\IfFileExists{microtype.sty}{%
\usepackage{microtype}
\UseMicrotypeSet[protrusion]{basicmath} % disable protrusion for tt fonts
}{}
\usepackage[margin = 1.5in]{geometry}
\usepackage{hyperref}
\PassOptionsToPackage{usenames,dvipsnames}{color} % color is loaded by hyperref
\hypersetup{unicode=true,
            pdftitle={Introduction to Basic R},
            pdfauthor={Abhinav Anand},
            colorlinks=true,
            linkcolor=blue,
            citecolor=magenta,
            urlcolor=red,
            breaklinks=true}
\urlstyle{same}  % don't use monospace font for urls
\usepackage{color}
\usepackage{fancyvrb}
\newcommand{\VerbBar}{|}
\newcommand{\VERB}{\Verb[commandchars=\\\{\}]}
\DefineVerbatimEnvironment{Highlighting}{Verbatim}{commandchars=\\\{\}}
% Add ',fontsize=\small' for more characters per line
\usepackage{framed}
\definecolor{shadecolor}{RGB}{248,248,248}
\newenvironment{Shaded}{\begin{snugshade}}{\end{snugshade}}
\newcommand{\KeywordTok}[1]{\textcolor[rgb]{0.13,0.29,0.53}{\textbf{#1}}}
\newcommand{\DataTypeTok}[1]{\textcolor[rgb]{0.13,0.29,0.53}{#1}}
\newcommand{\DecValTok}[1]{\textcolor[rgb]{0.00,0.00,0.81}{#1}}
\newcommand{\BaseNTok}[1]{\textcolor[rgb]{0.00,0.00,0.81}{#1}}
\newcommand{\FloatTok}[1]{\textcolor[rgb]{0.00,0.00,0.81}{#1}}
\newcommand{\ConstantTok}[1]{\textcolor[rgb]{0.00,0.00,0.00}{#1}}
\newcommand{\CharTok}[1]{\textcolor[rgb]{0.31,0.60,0.02}{#1}}
\newcommand{\SpecialCharTok}[1]{\textcolor[rgb]{0.00,0.00,0.00}{#1}}
\newcommand{\StringTok}[1]{\textcolor[rgb]{0.31,0.60,0.02}{#1}}
\newcommand{\VerbatimStringTok}[1]{\textcolor[rgb]{0.31,0.60,0.02}{#1}}
\newcommand{\SpecialStringTok}[1]{\textcolor[rgb]{0.31,0.60,0.02}{#1}}
\newcommand{\ImportTok}[1]{#1}
\newcommand{\CommentTok}[1]{\textcolor[rgb]{0.56,0.35,0.01}{\textit{#1}}}
\newcommand{\DocumentationTok}[1]{\textcolor[rgb]{0.56,0.35,0.01}{\textbf{\textit{#1}}}}
\newcommand{\AnnotationTok}[1]{\textcolor[rgb]{0.56,0.35,0.01}{\textbf{\textit{#1}}}}
\newcommand{\CommentVarTok}[1]{\textcolor[rgb]{0.56,0.35,0.01}{\textbf{\textit{#1}}}}
\newcommand{\OtherTok}[1]{\textcolor[rgb]{0.56,0.35,0.01}{#1}}
\newcommand{\FunctionTok}[1]{\textcolor[rgb]{0.00,0.00,0.00}{#1}}
\newcommand{\VariableTok}[1]{\textcolor[rgb]{0.00,0.00,0.00}{#1}}
\newcommand{\ControlFlowTok}[1]{\textcolor[rgb]{0.13,0.29,0.53}{\textbf{#1}}}
\newcommand{\OperatorTok}[1]{\textcolor[rgb]{0.81,0.36,0.00}{\textbf{#1}}}
\newcommand{\BuiltInTok}[1]{#1}
\newcommand{\ExtensionTok}[1]{#1}
\newcommand{\PreprocessorTok}[1]{\textcolor[rgb]{0.56,0.35,0.01}{\textit{#1}}}
\newcommand{\AttributeTok}[1]{\textcolor[rgb]{0.77,0.63,0.00}{#1}}
\newcommand{\RegionMarkerTok}[1]{#1}
\newcommand{\InformationTok}[1]{\textcolor[rgb]{0.56,0.35,0.01}{\textbf{\textit{#1}}}}
\newcommand{\WarningTok}[1]{\textcolor[rgb]{0.56,0.35,0.01}{\textbf{\textit{#1}}}}
\newcommand{\AlertTok}[1]{\textcolor[rgb]{0.94,0.16,0.16}{#1}}
\newcommand{\ErrorTok}[1]{\textcolor[rgb]{0.64,0.00,0.00}{\textbf{#1}}}
\newcommand{\NormalTok}[1]{#1}
\usepackage{graphicx,grffile}
\makeatletter
\def\maxwidth{\ifdim\Gin@nat@width>\linewidth\linewidth\else\Gin@nat@width\fi}
\def\maxheight{\ifdim\Gin@nat@height>\textheight\textheight\else\Gin@nat@height\fi}
\makeatother
% Scale images if necessary, so that they will not overflow the page
% margins by default, and it is still possible to overwrite the defaults
% using explicit options in \includegraphics[width, height, ...]{}
\setkeys{Gin}{width=\maxwidth,height=\maxheight,keepaspectratio}
\IfFileExists{parskip.sty}{%
\usepackage{parskip}
}{% else
\setlength{\parindent}{0pt}
\setlength{\parskip}{6pt plus 2pt minus 1pt}
}
\setlength{\emergencystretch}{3em}  % prevent overfull lines
\providecommand{\tightlist}{%
  \setlength{\itemsep}{0pt}\setlength{\parskip}{0pt}}
\setcounter{secnumdepth}{0}
% Redefines (sub)paragraphs to behave more like sections
\ifx\paragraph\undefined\else
\let\oldparagraph\paragraph
\renewcommand{\paragraph}[1]{\oldparagraph{#1}\mbox{}}
\fi
\ifx\subparagraph\undefined\else
\let\oldsubparagraph\subparagraph
\renewcommand{\subparagraph}[1]{\oldsubparagraph{#1}\mbox{}}
\fi

%%% Use protect on footnotes to avoid problems with footnotes in titles
\let\rmarkdownfootnote\footnote%
\def\footnote{\protect\rmarkdownfootnote}

%%% Change title format to be more compact
\usepackage{titling}

% Create subtitle command for use in maketitle
\newcommand{\subtitle}[1]{
  \posttitle{
    \begin{center}\large#1\end{center}
    }
}

\setlength{\droptitle}{-2em}
  \title{Introduction to Basic R}
  \pretitle{\vspace{\droptitle}\centering\huge}
  \posttitle{\par}
  \author{Abhinav Anand}
  \preauthor{\centering\large\emph}
  \postauthor{\par}
  \date{}
  \predate{}\postdate{}

\linespread{1.2}

\begin{document}
\maketitle

\section{Setup}\label{setup}

The following discussion assumes you have donwloaded R and RStudio. The
package \texttt{gapminder} needs to be installed prior to running the
commands below.

\begin{enumerate}
\def\labelenumi{\arabic{enumi}.}
\tightlist
\item
  For downloading R, visit \url{https://cran.r-project.org/}
\item
  For downloading RStudio visit \url{https://www.rstudio.com/}
\item
  To install, type \texttt{install.packages("gapminder")} in the RStudio
  console.
\end{enumerate}

The most basic way in which one can use R is as a calculator:

\begin{Shaded}
\begin{Highlighting}[]
\NormalTok{(t <-}\StringTok{ }\KeywordTok{sin}\NormalTok{(pi}\OperatorTok{/}\DecValTok{4}\NormalTok{)) }\CommentTok{#enclosing in parentheses prints output}
\end{Highlighting}
\end{Shaded}

\begin{verbatim}
## [1] 0.7071068
\end{verbatim}

\begin{Shaded}
\begin{Highlighting}[]
\NormalTok{(}\FloatTok{2.342929}\OperatorTok{*}\FloatTok{1.19483}\NormalTok{)}\OperatorTok{/}\FloatTok{4.9802244}
\end{Highlighting}
\end{Shaded}

\begin{verbatim}
## [1] 0.5621036
\end{verbatim}

\begin{Shaded}
\begin{Highlighting}[]
\NormalTok{x <-}\StringTok{ }\DecValTok{3}\OperatorTok{*}\FloatTok{10.4293939}
\KeywordTok{round}\NormalTok{(x, }\DataTypeTok{digits =} \DecValTok{4}\NormalTok{)}
\end{Highlighting}
\end{Shaded}

\begin{verbatim}
## [1] 31.2882
\end{verbatim}

\subsubsection{Remarks}\label{remarks}

Note the use of \texttt{\textless{}-} as opposed to \texttt{=}. In other
programming languages, one uses the equality sign for assignment but in
R the ``assignment operator'' is different. This is so because
\texttt{=} is reserved for use in function arguments. (Example:
\texttt{round(x,\ digits\ =\ 4)}. Note however, that even if \texttt{=}
is used in lieu of \texttt{\textless{}-} all codes work okay. Confirm by
computing, say \texttt{(x\ =\ 3*10.4293939)})

Some other useful operations:

\begin{Shaded}
\begin{Highlighting}[]
\DecValTok{1}\OperatorTok{:}\DecValTok{10}
\end{Highlighting}
\end{Shaded}

\begin{verbatim}
##  [1]  1  2  3  4  5  6  7  8  9 10
\end{verbatim}

\begin{Shaded}
\begin{Highlighting}[]
\KeywordTok{seq}\NormalTok{(}\DecValTok{1}\OperatorTok{:}\DecValTok{10}\NormalTok{) }\CommentTok{#the sequence generator function}
\end{Highlighting}
\end{Shaded}

\begin{verbatim}
##  [1]  1  2  3  4  5  6  7  8  9 10
\end{verbatim}

R has several packages with in-built functions. In order to use them,
one needs to declare their use. For our purposes, we will extensively
use the suite of packages \texttt{tidyverse} written primarily by Hadley
Wickham.

\begin{Shaded}
\begin{Highlighting}[]
\KeywordTok{library}\NormalTok{(tidyverse)}
\end{Highlighting}
\end{Shaded}

\section{Pipes}\label{pipes}

Consider the following problem: given a vector, we need to take its
sine, then take the mean, then square it and then if the resulting
number is more than 1, print ``more than 1'', else print ``less than
1''.

We can do this in the following way:

\begin{Shaded}
\begin{Highlighting}[]
\NormalTok{x <-}\StringTok{ }\DecValTok{1}\OperatorTok{:}\DecValTok{100} \CommentTok{#initial vector}

\NormalTok{x_}\DecValTok{1}\NormalTok{ <-}\StringTok{ }\KeywordTok{sin}\NormalTok{(x) }\CommentTok{#note vectorized function}

\NormalTok{x_}\DecValTok{2}\NormalTok{ <-}\StringTok{ }\NormalTok{(}\KeywordTok{mean}\NormalTok{(x_}\DecValTok{1}\NormalTok{))}\OperatorTok{^}\DecValTok{2} \CommentTok{#take mean, square}

\ControlFlowTok{if}\NormalTok{(x_}\DecValTok{2} \OperatorTok{>=}\StringTok{ }\DecValTok{1}\NormalTok{) }\CommentTok{#note the syntax for the if statement}
\NormalTok{\{}
  \KeywordTok{print}\NormalTok{(}\StringTok{"more than 1"}\NormalTok{)}
\NormalTok{\} }\ControlFlowTok{else}
\NormalTok{\{}
  \KeywordTok{print}\NormalTok{(}\StringTok{"less than 1"}\NormalTok{)}
\NormalTok{\}}
\end{Highlighting}
\end{Shaded}

\begin{verbatim}
## [1] "less than 1"
\end{verbatim}

\begin{Shaded}
\begin{Highlighting}[]
\CommentTok{# or more succinctly in one line}

\ControlFlowTok{if}\NormalTok{(}\KeywordTok{mean}\NormalTok{(}\KeywordTok{sin}\NormalTok{(x))}\OperatorTok{^}\DecValTok{2} \OperatorTok{>=}\StringTok{ }\DecValTok{1}\NormalTok{)}
\NormalTok{\{}
  \KeywordTok{print}\NormalTok{(}\StringTok{"more than 1"}\NormalTok{)}
\NormalTok{\} }\ControlFlowTok{else}
\NormalTok{\{}
  \KeywordTok{print}\NormalTok{(}\StringTok{"less than 1"}\NormalTok{)}
\NormalTok{\}}
\end{Highlighting}
\end{Shaded}

\begin{verbatim}
## [1] "less than 1"
\end{verbatim}

However, there is another way to do this with pipes which can be read as
the ``then'' operator. In fact, whenever we encounter the pipe symbol
\texttt{\%\textgreater{}\%}, without loss of generality, we can replace
it with the word ``then''.

\begin{Shaded}
\begin{Highlighting}[]
\NormalTok{x_}\DecValTok{3}\NormalTok{ <-}\StringTok{ }\NormalTok{x }\OperatorTok\StringTok{ }\KeywordTok{sin}\NormalTok{(.) }\OperatorTok\StringTok{ }\KeywordTok{mean}\NormalTok{(.) }\OperatorTok\StringTok{ }\NormalTok{.}\OperatorTok{^}\DecValTok{2} 
\CommentTok{# read as: take x, then take sine }
\CommentTok{# then take mean, then square}

\ControlFlowTok{if}\NormalTok{(x_}\DecValTok{3} \OperatorTok{>=}\StringTok{ }\DecValTok{1}\NormalTok{)}
\NormalTok{\{}
  \KeywordTok{print}\NormalTok{(}\StringTok{"more than 10"}\NormalTok{)}
\NormalTok{\} }\ControlFlowTok{else}
\NormalTok{\{}
  \KeywordTok{print}\NormalTok{(}\StringTok{"less than 1"}\NormalTok{)}
\NormalTok{\}}
\end{Highlighting}
\end{Shaded}

\begin{verbatim}
## [1] "less than 1"
\end{verbatim}

Pipes are very powerful when expressing a sequence of operations. The
pipe, \texttt{\%\textgreater{}\%} comes from the \texttt{magrittr}
package but the tidyverse loads it automatically.

\section{Data-types in R: Some
Examples}\label{data-types-in-r-some-examples}

\subsection{Vectors}\label{vectors}

There are two main types of vectors in R: atomic and non-atomic. Atomic
vectors have components of the same type, say double, or integer or
logical etc. These are also referred to as ``numeric'' vectors. By
default all vectors in R are considered column-vectors.

Non-atomic vectors (including ``lists'') can contain heterogenous
components.

\subsubsection{Recycling}\label{recycling}

There are no built-in scalars in R. Scalars are implemented as vectors
of length 1. Most functions in R, hence, are vectorized---they take
vector arguments and operate on it component-wise. For example, for the
vector \texttt{x\ \textless{}-\ 1:100}, the function \texttt{sin(x)}
produces each component's sine, i.e., \texttt{sin(1):sin(100)}.

When we mix scalars and vectors, the scalars are automatically
replicated to be the same size as the vector. For example:

\begin{Shaded}
\begin{Highlighting}[]
\DecValTok{1} \OperatorTok{+}\StringTok{ }\DecValTok{1}\OperatorTok{:}\DecValTok{10}
\end{Highlighting}
\end{Shaded}

\begin{verbatim}
##  [1]  2  3  4  5  6  7  8  9 10 11
\end{verbatim}

\begin{Shaded}
\begin{Highlighting}[]
\CommentTok{#is the same operation as }

\KeywordTok{rep}\NormalTok{(}\DecValTok{1}\NormalTok{, }\DecValTok{10}\NormalTok{) }\OperatorTok{+}\StringTok{ }\DecValTok{1}\OperatorTok{:}\DecValTok{10} \CommentTok{#note the highly useful rep() function}
\end{Highlighting}
\end{Shaded}

\begin{verbatim}
##  [1]  2  3  4  5  6  7  8  9 10 11
\end{verbatim}

A related question: what happens if we add two vectors of different
lengths?

\begin{Shaded}
\begin{Highlighting}[]
\DecValTok{1}\OperatorTok{:}\DecValTok{3} \OperatorTok{+}\StringTok{ }\DecValTok{1}\OperatorTok{:}\DecValTok{15}
\end{Highlighting}
\end{Shaded}

\begin{verbatim}
##  [1]  2  4  6  5  7  9  8 10 12 11 13 15 14 16 18
\end{verbatim}

\begin{Shaded}
\begin{Highlighting}[]
\CommentTok{#is the same as}

\KeywordTok{rep}\NormalTok{(}\DecValTok{1}\OperatorTok{:}\DecValTok{3}\NormalTok{, }\DecValTok{5}\NormalTok{) }\OperatorTok{+}\StringTok{ }\DecValTok{1}\OperatorTok{:}\DecValTok{15}
\end{Highlighting}
\end{Shaded}

\begin{verbatim}
##  [1]  2  4  6  5  7  9  8 10 12 11 13 15 14 16 18
\end{verbatim}

This is ``recycling'': R replicates the shorter vector to the same
length as the longer vector and then adds the two together. While such
usage is uncommon for other programming languages, it has undeniable
utility, though one needs to be cautious when implementing this idea.

\subsubsection{Note}\label{note}

\begin{enumerate}
\def\labelenumi{\arabic{enumi}.}
\tightlist
\item
  While recycling works for vectors, it doesn't do so for matrices or
  other rectangular data-types.
\end{enumerate}

\subsubsection{\texorpdfstring{The \texttt{c()}
operator}{The c() operator}}\label{the-c-operator}

The \texttt{c()} operator stands for ``concatenate'' or according to
some writers, ``combine''.

\begin{Shaded}
\begin{Highlighting}[]
\NormalTok{y <-}\StringTok{ }\DecValTok{1}\OperatorTok{:}\DecValTok{4}

\NormalTok{(y_}\DecValTok{2}\NormalTok{ <-}\StringTok{ }\KeywordTok{c}\NormalTok{(}\DecValTok{5}\OperatorTok{:}\DecValTok{10}\NormalTok{, y)) }\CommentTok{#concatenate}
\end{Highlighting}
\end{Shaded}

\begin{verbatim}
##  [1]  5  6  7  8  9 10  1  2  3  4
\end{verbatim}

\begin{Shaded}
\begin{Highlighting}[]
\NormalTok{(y_}\DecValTok{3}\NormalTok{ <-}\StringTok{ }\NormalTok{y_}\DecValTok{2}\NormalTok{[}\KeywordTok{c}\NormalTok{(}\DecValTok{3}\OperatorTok{:}\DecValTok{7}\NormalTok{)] )}\CommentTok{#note the square brackets}
\end{Highlighting}
\end{Shaded}

\begin{verbatim}
## [1]  7  8  9 10  1
\end{verbatim}

\begin{Shaded}
\begin{Highlighting}[]
\NormalTok{(y_}\DecValTok{4}\NormalTok{ <-}\StringTok{ }\NormalTok{y_}\DecValTok{2}\NormalTok{[}\OperatorTok{-}\KeywordTok{c}\NormalTok{(}\DecValTok{1}\NormalTok{, }\DecValTok{4}\NormalTok{)]) }\CommentTok{#note the - sign}
\end{Highlighting}
\end{Shaded}

\begin{verbatim}
## [1]  6  7  9 10  1  2  3  4
\end{verbatim}

\subsection{Arrays and Matrices}\label{arrays-and-matrices}

Matrices are two-dimensional data-types with columns and rows as the two
dimensions. Arrays are data-types that can contain more than two
dimensions as well (height, say in addition to rows and columns). These
may or may not have special named attributes such as column names or row
names.

Matrices can be constructed from component vectors by the usage of
commands \texttt{cbind()} and \texttt{rbind()}

\begin{Shaded}
\begin{Highlighting}[]
\NormalTok{c_}\DecValTok{1}\NormalTok{ <-}\StringTok{ }\KeywordTok{c}\NormalTok{(}\DecValTok{4}\NormalTok{, }\DecValTok{9}\NormalTok{, }\DecValTok{10}\NormalTok{, }\DecValTok{12}\NormalTok{)}
\NormalTok{c_}\DecValTok{2}\NormalTok{ <-}\StringTok{ }\KeywordTok{c}\NormalTok{(}\DecValTok{10}\NormalTok{, }\DecValTok{3}\NormalTok{, }\DecValTok{1}\NormalTok{, }\DecValTok{10}\NormalTok{)}

\NormalTok{(mat_c <-}\StringTok{ }\KeywordTok{cbind}\NormalTok{(c_}\DecValTok{1}\NormalTok{, c_}\DecValTok{2}\NormalTok{)) }\CommentTok{#column-bind}
\end{Highlighting}
\end{Shaded}

\begin{verbatim}
##      c_1 c_2
## [1,]   4  10
## [2,]   9   3
## [3,]  10   1
## [4,]  12  10
\end{verbatim}

\begin{Shaded}
\begin{Highlighting}[]
\NormalTok{(mat_r <-}\StringTok{ }\KeywordTok{rbind}\NormalTok{(c_}\DecValTok{1}\NormalTok{, c_}\DecValTok{2}\NormalTok{)) }\CommentTok{#row-bind}
\end{Highlighting}
\end{Shaded}

\begin{verbatim}
##     [,1] [,2] [,3] [,4]
## c_1    4    9   10   12
## c_2   10    3    1   10
\end{verbatim}

\subsection{Dataframes}\label{dataframes}

Dataframes are data-types with a collection of vectors of the same
length which may be of different types. Each column (variable) has a
column-name that can be used to access the whole vector. Additionally,
the column can be extracted by appending to the dataframe, the
\texttt{\$} sign followed by the column-name. The full set of names can
be extracted by the command \texttt{names()}

\begin{Shaded}
\begin{Highlighting}[]
\NormalTok{df <-}\StringTok{ }\KeywordTok{data.frame}\NormalTok{(}\DataTypeTok{a =} \KeywordTok{runif}\NormalTok{(}\DecValTok{1}\OperatorTok{:}\DecValTok{10}\NormalTok{), }
                 \DataTypeTok{b =} \KeywordTok{rnorm}\NormalTok{(}\DecValTok{10}\NormalTok{, }\DecValTok{0}\NormalTok{, }\DecValTok{1}\NormalTok{),}
                 \DataTypeTok{c =} \DecValTok{11}\OperatorTok{:}\DecValTok{20}
\NormalTok{                 )}

\NormalTok{df}\OperatorTok{$}\NormalTok{a }\OperatorTok\StringTok{ }\KeywordTok{head}\NormalTok{()}
\end{Highlighting}
\end{Shaded}

\begin{verbatim}
## [1] 0.9614592 0.9158706 0.8321787 0.1612111 0.2023648 0.4352416
\end{verbatim}

\begin{Shaded}
\begin{Highlighting}[]
\KeywordTok{names}\NormalTok{(df)}
\end{Highlighting}
\end{Shaded}

\begin{verbatim}
## [1] "a" "b" "c"
\end{verbatim}

\subsubsection{Subsetting Dataframes}\label{subsetting-dataframes}

If we need only part of a dataframe, we can refer to the relevant
indices in square brackets appended to the name of the dataframe.
Leaving an index blank includes all entries.

\begin{Shaded}
\begin{Highlighting}[]
\NormalTok{df[, }\DecValTok{1}\NormalTok{] }\OperatorTok\StringTok{ }\KeywordTok{head}\NormalTok{() }\CommentTok{#column 1}
\end{Highlighting}
\end{Shaded}

\begin{verbatim}
## [1] 0.9614592 0.9158706 0.8321787 0.1612111 0.2023648 0.4352416
\end{verbatim}

\begin{Shaded}
\begin{Highlighting}[]
\NormalTok{df[}\DecValTok{3}\NormalTok{, ] }\OperatorTok\StringTok{ }\KeywordTok{head}\NormalTok{() }\CommentTok{#row 3}
\end{Highlighting}
\end{Shaded}

\begin{verbatim}
##           a        b  c
## 3 0.8321787 2.206687 13
\end{verbatim}

\begin{Shaded}
\begin{Highlighting}[]
\NormalTok{df[}\KeywordTok{c}\NormalTok{(}\DecValTok{1}\NormalTok{,}\DecValTok{2}\NormalTok{), }\StringTok{"c"}\NormalTok{]}
\end{Highlighting}
\end{Shaded}

\begin{verbatim}
## [1] 11 12
\end{verbatim}

\section{Functions}\label{functions}

Functions automate common tasks succinctly. According to Hadley Wickham,
a general rule of thumb is that if we need to copy-paste more than
thrice, it's time to write a function. Writing functions has many uses:
its name betrays its intentions; if changes are needed only one edit is
enough; and the chance of errors while copy-pasting decreases
drastically when working with functions.

Suppose there are normal random variables and we wish to transform them
to standard normal.

\begin{Shaded}
\begin{Highlighting}[]
\NormalTok{t_}\DecValTok{1}\NormalTok{ <-}\StringTok{ }\KeywordTok{rnorm}\NormalTok{(}\DecValTok{100}\NormalTok{, }\DecValTok{5}\NormalTok{, }\DecValTok{10}\NormalTok{) }\CommentTok{#number of points, mean, sigma}
\NormalTok{t_}\DecValTok{2}\NormalTok{ <-}\StringTok{ }\KeywordTok{rnorm}\NormalTok{(}\DecValTok{100}\NormalTok{, }\DecValTok{10}\NormalTok{, }\DecValTok{100}\NormalTok{)}
\NormalTok{t_}\DecValTok{3}\NormalTok{ <-}\StringTok{ }\KeywordTok{rnorm}\NormalTok{(}\DecValTok{100}\NormalTok{, }\DecValTok{20}\NormalTok{, }\DecValTok{200}\NormalTok{)}

\NormalTok{t_1_s <-}\StringTok{ }\NormalTok{(t_}\DecValTok{1}\OperatorTok{-}\DecValTok{5}\NormalTok{)}\OperatorTok{/}\DecValTok{10} 
\NormalTok{t_2_s <-}\StringTok{ }\NormalTok{(t_}\DecValTok{1}\OperatorTok{-}\DecValTok{10}\NormalTok{)}\OperatorTok{/}\DecValTok{100}
\NormalTok{t_3_s <-}\StringTok{ }\NormalTok{(t_}\DecValTok{1}\OperatorTok{-}\DecValTok{20}\NormalTok{)}\OperatorTok{/}\DecValTok{200}
\end{Highlighting}
\end{Shaded}

However, this could be automated if we write a function

\begin{Shaded}
\begin{Highlighting}[]
\NormalTok{norm_std <-}\StringTok{ }\ControlFlowTok{function}\NormalTok{(t, mu, sigma) }\CommentTok{#note the syntax}
\NormalTok{\{}
\NormalTok{  t_std <-}\StringTok{ }\NormalTok{(t }\OperatorTok{-}\StringTok{ }\NormalTok{mu)}\OperatorTok{/}\NormalTok{sigma}
  
  \KeywordTok{return}\NormalTok{(t_std) }\CommentTok{#note the return function}
\NormalTok{\}}

\KeywordTok{norm_std}\NormalTok{(t_}\DecValTok{1}\NormalTok{, }\DecValTok{5}\NormalTok{, }\DecValTok{10}\NormalTok{) }\OperatorTok\StringTok{ }\KeywordTok{head}\NormalTok{(.) }\CommentTok{#what does this mean?}
\end{Highlighting}
\end{Shaded}

\begin{verbatim}
## [1]  0.4755169 -0.5812264  1.8950885 -0.7009385  0.4171631  0.7818750
\end{verbatim}

\begin{Shaded}
\begin{Highlighting}[]
\KeywordTok{norm_std}\NormalTok{(t_}\DecValTok{2}\NormalTok{, }\DecValTok{10}\NormalTok{, }\DecValTok{100}\NormalTok{) }\OperatorTok\StringTok{ }\KeywordTok{head}\NormalTok{(.)}
\end{Highlighting}
\end{Shaded}

\begin{verbatim}
## [1]  0.36904049  0.52473282 -1.06526400 -0.02148587 -0.28236561 -0.77917243
\end{verbatim}

\begin{Shaded}
\begin{Highlighting}[]
\KeywordTok{norm_std}\NormalTok{(t_}\DecValTok{3}\NormalTok{, }\DecValTok{20}\NormalTok{, }\DecValTok{200}\NormalTok{) }\OperatorTok\StringTok{ }\KeywordTok{head}\NormalTok{(.)}
\end{Highlighting}
\end{Shaded}

\begin{verbatim}
## [1] -1.90869015 -1.44342007 -0.24974682 -0.02721761 -1.10288651 -0.34579283
\end{verbatim}

\section{Directory Management}\label{directory-management}

If we save something in R or if we wish to access some file in R, we
need to ensure that they are in the same folder (directory) as our code.
This gives rise to the notion of the working directory which is
displayed at the top of the console. We can explicitly find the name of
the working directory by the command \texttt{getwd()}.

To pinpoint addresses of folders, Mac and Linux use slashes (say
\texttt{plots/plot\_1.pdf}) but Windows uses backslashes
(\texttt{plots\textbackslash{}plot\_1.pdf}). However, backslashes are
reserved in R---in that they cannot be used as they are. Hence to
include each backslash, we need to prefix another backslash, which may
be confusing. Hence many writers suggest using the Linux/Mac style
addresses.

\section{RStudio Projects}\label{rstudio-projects}

It is good practice to keep all files related to a project together in
one folder. This may include script files (with .R extension), plots,
data files etc. RStudio uses ``projects'' with .Rproj extension for this
special purpose.

In general, it is considered good practice to keep one project for each
data analysis exercise and put all data files, scripts, outputs (such as
plots) in that folder. It is also recommended that relative paths be
used and not absolute paths.


\end{document}
